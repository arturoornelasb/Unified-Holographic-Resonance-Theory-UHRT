\documentclass[11pt, a4paper]{article}

% --- ESSENTIAL PACKAGES ---
\usepackage[english]{babel} 
\usepackage[T1]{fontenc} 
\usepackage{geometry} 
\geometry{a4paper, top=2.5cm, bottom=2.5cm, left=2cm, right=2cm}

% --- MATHEMATICAL PACKAGES ---
\usepackage{amsmath} 
\usepackage{amsfonts} 
\usepackage{amssymb} 
\usepackage{graphicx} 

% --- ADDITIONAL PACKAGES ---
\usepackage{url}
\usepackage{hyperref} 
\hypersetup{
	colorlinks=true,
	linkcolor=blue,
	filecolor=magenta,
	urlcolor=cyan,
	pdftitle={Composite Dimensional Evolution Framework: A Foundational Theory based on Arithmetic Proportionality},
	pdfauthor={José Arturo Ornelas Brand},
	pdfkeywords={Dimensional emergence, Arithmetic proportionality, Quantum glitch, Holographic principle, Pyramidal topology}
}
\usepackage{bookmark} 
\usepackage{enumitem} 

% --- TITLE AND METADATA ---
\title{\textbf{Composite Dimensional Evolution Framework:}\\A Foundational Theory based on Arithmetic Proportionality}
\author{
	José Arturo Ornelas Brand \\
	\textit{Independent Researcher} \\
	\href{mailto:arturoornelas62@gmail.com}{\texttt{arturoornelas62@gmail.com}}
}
\date{\today}

\begin{document}
	
	\maketitle
	\tableofcontents 
	\newpage
	
	% --- ABSTRACT ---
	\begin{abstract}
		This document presents a theoretical framework that postulates a fundamental isomorphism between geometric dimensional progression and the evolution of arithmetic proportionality. We propose that the \textbf{Simple Rule of Three} is not merely a calculation tool, but the "Theory of Everything" (ToE) of a flat (2D) universe. We argue that higher dimensions emerge through a process of "composite duplication" that is algebraically reflected in the \textbf{Compound Rule of Three}.
		
		We introduce the concept of the \textbf{"Glitch"} as a quantum catalyst (analogous to the collapse of a qubit) that forces the dimensional leap. Finally, we model this progression as a \textbf{pyramidal structure} with a "Double Composite Echo," where the "reality density" at each level is measured using \textbf{inverse proportional relationships}.
	\end{abstract}
	
	\vspace{5mm}
	\noindent
	\textbf{Keywords:} Dimensional emergence, Arithmetic proportionality, Quantum glitch, Holographic principle, Pyramidal topology, Symmetry breaking, Rule of Three
	
	% --- SECTION 1 ---
	\section{Origins and Basic Dimensional Progression}
	
	The inspiration for this framework comes from the architectural and geometric principles of minimal form \cite{kandinsky1926}. It begins with the simplest space and evolves step by step.
	
	\subsection{Step 1: 0D (The Point)}
	\begin{itemize}
		\item \textbf{Geometry:} A point with no extension.
		\item \textbf{Concept:} The minimal unit of existence. It is a fundamental binary state: it exists (1), or it does not (0).
		\item \textbf{Mathematics:} A scalar. A constant, $k$.
		\item \textbf{Limit:} Rigid and absolute.
	\end{itemize}
	
	\subsection{Step 2: 1D (The Line)}
	\begin{itemize}
		\item \textbf{Geometry:} Simple duplication of the point (Point $A \to$ Point $B$).
		\item \textbf{Concept:} Introduces \textit{extension} and \textit{direction}.
		\item \textbf{Mathematics:} The Simple Ratio. The relationship between two quantities, $\frac{A}{B}$. This is the fundamental "slope," the basic rate of change.
		\item \textbf{Limit:} Confined to a straight line.
	\end{itemize}
	
	\subsection{Step 3: 2D (The Plane)}
	\begin{itemize}
		\item \textbf{Geometry:} Composite duplication of the line (perpendicular extrusion).
		\item \textbf{Concept:} Introduces \textit{area}, \textit{angles}, and basic topology. It is the first dimension where structural stability (e.g., a triangle) is possible.
		\item \textbf{Mathematics:} The \textbf{Simple Rule of Three}. It is the "ToE" of this flat universe, as it stabilizes the system by equating two 1D ratios. It allows for prediction and order.
		\begin{equation}
			\frac{A}{B} = \frac{C}{X} \implies X = \frac{B \cdot C}{A}
		\end{equation}
		\item \textbf{Limit:} Planar, with no depth.
	\end{itemize}
	
	\subsection{Step 4: 3D (The Volume)}
	\begin{itemize}
		\item \textbf{Geometry:} Composite duplication of the plane (perpendicular extrusion).
		\item \textbf{Concept:} Introduces \textit{volume}, \textit{chirality} (handedness), and \textit{knots}.
		\item \textbf{Mathematics:} The \textbf{Compound Rule of Three}. It is the "ToE" of 3D. It takes the 2D law and multiplies it by another 1D ratio (a new independent variable, $Z$).
		\begin{equation}
			\text{Result} \propto \text{Variable}_X \cdot \text{Variable}_Y \cdot \text{Variable}_Z
		\end{equation}
		The formula to solve for $X$ involves a multiplication of ratios:
		\begin{equation}
			\left( \frac{A_1}{B_1} \right) \cdot \left( \frac{A_2}{B_2} \right) = \frac{C_{\text{final}}}{X_{\text{final}}}
		\end{equation}
		\item \textbf{Limit:} Three-dimensional volumetric.
	\end{itemize}
	
	\subsection{Step 5: nD (The Hyperspace)}
	\begin{itemize}
		\item \textbf{Geometry:} Recursive extrusion of the $(n-1)D$ space.
		\item \textbf{Concept:} Hypervolumes, n-axis complexity.
		\item \textbf{Mathematics:} The generalization of the Compound Rule of Three. The result is proportional to the product of $n-1$ independent variables.
		\begin{equation}
			\left( \prod_{i=1}^{n-1} \frac{A_i}{B_i} \right) = \frac{C}{X}
		\end{equation}
	\end{itemize}
	
	% --- SECTION 2 ---
	\section{The "Glitch": Catalyst for the Dimensional Leap}
	
	The progression is not passive. A simple duplication would only create a larger version of the same dimension. To generate a \textit{new} (orthogonal) dimension, a catalyst is needed.
	
	\subsection{The Glitch as Symmetry Breaking}
	We define the \textbf{"Glitch"} as a disruptive event, a bifurcation, or a spontaneous symmetry breaking. It is a moment of instability where the system, to stabilize itself, must "choose" a new direction of expansion, thus creating the next dimensional level.
	
	\subsection{The Quantum Bridge: The Qubit as Origin}
	The "Glitch" has a perfect quantum analog:
	\begin{itemize}
		\item \textbf{0D as Potential:} The qubit in superposition ($\alpha|0\rangle + \beta|1\rangle$) represents the 0D state. It is neither 0 nor 1; it is the potential for both.
		\item \textbf{The Glitch as Collapse:} The act of measurement (or an interaction) is the "Glitch" that collapses the wave function and forces the system to "exist" in a state (0 or 1), initiating the 1D progression.
		\item \textbf{Duplication as Entanglement:} The "composite duplication" of the qubit is entanglement. $n$ entangled qubits can encode $2^n$ states, an exponential growth that mirrors the expansion of our pyramid.
	\end{itemize}
	The Big Bang can be interpreted as the initial "Glitch."
	
	% --- SECTION 3 ---
	\section{Pyramidal Structure and Measurement}
	
	\subsection{The Pyramid of Double Composite Echo}
	The topology of the system is a pyramid:
	\begin{itemize}
		\item \textbf{Apex (0D):} The source, the primordial qubit.
		\item \textbf{Layers (nD):} Each dimensional level is the base for the next.
		\item \textbf{Composite Echo:} Each layer $n$ is an "echo" of layer $n-1$, but compounded (multiplied, $k^n$).
		\item \textbf{Double:} The term "double" refers to the quantum duality ($\alpha|0\rangle, \beta|1\rangle$) that reverberates in each layer.
	\end{itemize}
	
	\subsection{The Bidirectional Bridge (Holography)}
	The framework allows for a two-way flow of information:
	\begin{enumerate}
		\item \textbf{Ascent (Bottom-Up):} The evolution we have described, generating complexity.
		\item \textbf{Descent (Top-Down):} The projection. A Compound Rule of Three (3D) can be "collapsed" or simplified to a Simple Rule of Three (2D) if one of the variables is held constant. This is analogous to the holographic principle, where the information of an $n$D volume can be encoded on its $(n-1)D$ boundary \cite{thooft1993,susskind1995}.
	\end{enumerate}
	
	\subsection{Measurement of Dimensional Density}
	If we assume a finite amount of "Existence" or "Energy" ($E_k$) distributed throughout the pyramid, we must use an inverse relationship to measure the "density" of each layer.
	
	The generalized "Volume" (Lebesgue measure) of an $n$-dimension scales as $L^n$. Therefore, the Density ($\rho$) is:
	\begin{equation}
		\rho_n = \frac{E_k}{V_n} \propto \frac{1}{L^n}
	\end{equation}
	This is governed by an \textbf{Inverse Rule of Three}:
	\begin{equation}
		\rho_A \cdot V_A = \rho_B \cdot V_B
	\end{equation}
	\textbf{Implication:} Higher dimensions are intrinsically more "diluted." This explains the apparent stability of our 3D universe, where forces (e.g., gravity $\propto 1/r^2$) are not diluted as much as they would be in higher dimensions ($\propto 1/r^{n-1}$).
	
	% --- SECTION 4 ---
	\section{Conclusion: Towards an Arithmetic ToE}
	The Composite Dimensional Evolution Framework offers an interdisciplinary lens (architecture, arithmetic, quantum physics) to reinterpret the structure of reality.
	
	The originality lies not in the individual components (dimensional analysis, proportionality), but in the \textit{framing}: the Rule of Three as a \textbf{generative and evolutionary principle}, with the quantum Glitch as its engine. This framework could inspire new pedagogical models, architectural designs, and even alternative cosmologies based on the primacy of arithmetic proportionality.
	
	\begin{thebibliography}{9}
		
		\bibitem{kandinsky1926}
		Wassily Kandinsky. \textit{Point and Line to Plane}. Bauhausbücher Vol. 9, Bauhaus Books, 1926. \url{https://www.lars-mueller-publishers.com/point-and-line-plane}.
		
		\bibitem{thooft1993}
		Gerard 't Hooft. Dimensional Reduction in Quantum Gravity. \textit{arXiv preprint gr-qc/9310026}, 1993. Published in \textit{Salamfestschrift: A Collection of Talks}, World Scientific, pp. 284-296, 1994. DOI: \url{10.1142/9789814535168_0028}. \url{https://arxiv.org/abs/gr-qc/9310026}.
		
		\bibitem{susskind1995}
		Leonard Susskind. The World as a Hologram. \textit{Journal of Mathematical Physics}, 36(11):6377-6396, 1995. DOI: \url{10.1063/1.531249}. \url{https://arxiv.org/abs/hep-th/9409089}.
	\end{thebibliography}
	
\end{document}